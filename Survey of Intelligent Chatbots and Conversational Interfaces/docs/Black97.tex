\subsection{Unit Selection in Speech Synthesis \cite{Black97}}

This paper describes a new method for synthesizing speech by concatenating sub-word units from a database of labelled speech. The method is implemented within a full text-to-speech system, called the \emph{Festival}, offering efficient natural sounding speech synthesis.

In the selection based synthesis, there is a large database of speech with a variable number of units from a particular class. The goal of these algorithms is to select the best sequence of units from all the possibilities in the database, and concatenate them to produce the final speech.

The basic idea of the unit selection technique proposed in the paper is summarized as follows: 1) A large units inventory is created by automatically clustering units of the same phone class based on their phonetic and prosodic context; 2) The appropriate cluster is then selected for a target unit; 3) Finally an optimal path is found through the candidate units based on their distance from the cluster center and an acoustically based join cost.

The first step is to cluster the units. The paper defines an acoustic measure to measure the distance between two units of the same phone type, by using an acoustic vector which comprises Mel frequency cepstrum coefficients, $F_0$, power, etc. The acoustic distance between two units is simply the average distance for the vectors (including some frames in the previous units). The distance measure is used by the CART algorithm, which builds a decision tree whose questions best minimise the impurity of the sub-clusters.

To join consecutive candidate units from clusters selected by the decision trees, it uses an optimal coupling technique to measure the concatenation costs between two units. This technique offers two results: the cost of a join, and a position for the join. Optimal coupling allows it to select more stable positions towards the center of the phone.

At synthesis time the input is a stream of target segments to be synthesized. For each target it uses the CART for that unit type, and ask the questions to find the  appropriate cluster which provides a set of candidate units. Let $Tdist(U)$ be the distance of a unit $U$ to its cluster center, and the function $Jdist(U_i, U_{i-1})$ be the join cost. The proposed method uses a Viterbi search to find the optimal path through the candidate units that minimizes the following expression:
$$\sum_{i=1}^N Tdist(U_i) + W * Jdist(U_i, U_{i-1}).$$

The paper further discusses the pruning technique, which can reduce the size of the distributed database. Pruning has two effects: 1) remove spurious atypical units which may have been caused by mislabelling or poor articulation in the original recording; 2) remove those units which are so common that there is no significant distinction.

In the experimental study, the paper tries the proposed method with two databases: 1) TIMIT, a male British English speaker (about 14,000 units); 2) Boston University FM Radio corpus, a female American news reader (about 37,000 units). In the test it generated 20 sentences for each evaluated model. The sentences from each model were played against those from another set, and the subjects decide which ones are better. In this way, the author conducted comparison study by varying cluster size, $F_0$ weight in the acoustic cost, and the amount to prune final clusters. The empirically optimal parameters were respectively reported.
